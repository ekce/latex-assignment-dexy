\newcommand{\preamblefolder}{./include}
\RequirePackage[l2tabu, orthodox]{nag}        % Checks for obsolete habits and stuff, must be called first.
%\RequirePackage{fixltx2e}                     % Implements LaTeX2 fixes.

% Document class and typography packages
\documentclass[letterpaper,DIV=20]{scrartcl}  % KOMA-Script article class, letterpaper option sets page size and DIV sets margins (remove DIV option if problems arise).
\setkomafont{disposition}{\normalfont}        % Sets KOMA-Script title font.
%\usepackage[margin=1in]{geometry}             % Sets the margins to a full size page, margin=1.5cm is also common but cuts off page numbers.
\usepackage{microtype}                        % Nicer kerning.


% Input encoding and standards packages
\usepackage{isomath}                          % Makes math conform to ISO 80000-2 mathematical style. Includes fixmath and kvoptions packages.
\usepackage[utf8]{inputenc}                   % Sets the input encoding to UTF-8.


% Font packages and commands
\usepackage[sc,osf]{mathpazo}                 % Provides Palatino typeface, options: sc = smallcaps, osf = oldstyle numerals. (rm and math)
\linespread{1.05}                             % Changes the vertical spacing between lines as recommended for Palatino fonts.
\AtBeginDocument{                             % Provides AMS mathbb fonts.
  \DeclareSymbolFont{AMSb}{U}{msb}{m}{n}
  \DeclareSymbolFontAlphabet{\mathbb}{AMSb}
}
\usepackage[scaled]{helvet}                   % Helvetica (ss)
%\usepackage{classico}                        % Classico (ss), broken
%\usepackage{courier}                         % Courier (tt)
%\renewcommand{\ttdefault}{lmtt}              % Latin Modern (tt)
\usepackage[lf]{sourcecodepro}                % Source Code Pro (tt)
\usepackage{mathrsfs}                         % Support for script (\mathscr) fonts.
\usepackage{dtklogos}                         % Provides logos like \LaTex.
\normalfont                                   % necessary to make LaTeX aware of our choice of the base font
\usepackage[T1]{fontenc}                      % Tells LaTeX to use T1 encoded version of fonts when available


% AMS math packages
\usepackage{amsmath}                          % AMS math.
\usepackage{amsthm}                           % AMS theorem.
\usepackage{amssymb}                          % AMS mathbb and other stuff.
\usepackage{mathtools}                        % Extension to amsmath, provides \DeclarePairedDelimiterX.


% Functionality providing packages
%\usepackage{scrextend}                       % KOMA-Script package, provides addmargin environment, not necessary if using KOMA-Script article class.
\usepackage{color}                            % Provides text-mode only text coloring.
\usepackage{cancel}                           % Provides \cancel for crossing out other symbols.
\usepackage{esvect}                           % Lets you typeset vector arrows with \vv{AB}.
\usepackage[retainorgcmds]{IEEEtrantools}     % IEEEeqnarray environment.
\usepackage{wrapfig}                          % Provides environment to let text wrap around figures.
\usepackage{fancyvrb}                         % Fancy verbatim environment (used by dexy)
\usepackage{enumitem}                         % Fancy enumerate environment.
\usepackage{array}                            % Fancy array environment.
\usepackage[pdfusetitle]{hyperref}            % Provides support for hyperlinks.
\usepackage{amsrefs}                          % Provides an environment for handling references and citations, needs to be loaded after hyperref.
\usepackage{bussproofs}                       % Provides environments for writing Gentzen style proofs.
\usepackage{float}                            % Provides float positioning for images (via \begin{figure}[H])
 %main packages
\usepackage{tikz} %tikz graphics
\usetikzlibrary{cd}
%\usetikzlibrary{arrows,automata,shapes}
 %tikz packages
% Topic specific commands
	% Set Theory
		\DeclarePairedDelimiterX{\card}[1]{\lvert}{\rvert}{#1}            % Cardinality function, depends on mathtools package.
		\DeclareMathOperator{\pow}{\mathcal{P}}                           % Power set function.
	% Graph Theory
		\DeclareMathOperator{\gver}{V}                                    % Vertex set function.
		\DeclareMathOperator{\gedg}{E}                                    % Edge set function.
		\DeclareMathOperator{\gdis}{d}                                    % Distance function.
		\DeclareMathOperator{\gdeg}{deg}                                  % Degree function.
		\DeclareMathOperator{\gclo}{C}                                    % Closure function.
	% Topology
		\DeclareMathOperator{\tn}{N}                                      % Neighborhood function.
		\DeclareMathOperator{\tbd}{bd}                                    % Set of Boundary points function.
		\DeclareMathOperator{\tint}{int}                                  % Set of Interior points function.
		\DeclareMathOperator{\tacc}{acc}                                  % Set of Accumulation points function.
		\DeclareMathOperator{\tiso}{iso}                                  % Set of Isolated points function.
		\DeclareMathOperator{\tcl}{cl}                                    % Closure function.
		\DeclareMathOperator{\sgn}{sgn}                                   % Signum function.
		\DeclareMathOperator{\dis}{dis}                                   % Discrete topology function.
		\DeclareMathOperator{\indis}{indis}                               % Indiscrete topology function.
		\DeclareMathOperator{\dist}{d}                                    % Distance (metric) function.
		\DeclareMathOperator{\der}{Der}                                   % Derived set function.
	% Analysis
		\newcommand*\Heq{\ensuremath{\overset{\kern2pt l'H}{=}}}          % An equal sign with l'H written above it for use with l'hopital theorem.
		\newcommand{\ud}{\,\mathrm{d}}                                    % Upright d with a space, for calculus infinitesimals.
		\DeclarePairedDelimiterX{\norm}[1]{\lVert}{\rVert}{#1}            % Norm function, depends on mathtools package.
		\renewcommand*{\Vec}[1]{\vv{\mathbf{#1}}}                         % Bold vectors with overline, depends on esvect package.
	% Number Theory
		\DeclareMathOperator{\lcm}{lcm}                                   % Lowest Common Multiple function.
	% Computability
		\newcommand{\monus}{\stackrel{{}^{\scriptstyle .}}{\smash{-}}}    % Monus operator.
		\DeclareMathOperator{\lt}{lt}                                     % Length function.
		\DeclareMathOperator{\halted}{halted}                             % Halted function.
	% Category Theory
		\newcommand{\catname}[1]{{\normalfont\textbf{#1}}}
		\newcommand{\Set}{\catname{Set}}
		\newcommand{\Rel}{\catname{Rel}}
		\newcommand{\Poset}{\catname{Poset}}


% Theorems, definitions, lemmas, and such.
\newtheorem*{thm}{Theorem}                                                % Defines theorem environment.
\newtheorem*{defn}{Definition}                                            % Defines definition environment.
\newtheorem*{cor}{Corollary}                                              % Defines corollary environment.
\newtheorem*{lem}{Lemma}                                                  % Defines lemma environment.
%\newtheorem*{ore}{Ore's Theorem}                                         % Example named theorem environment.


% Special symbols/commands
\newcommand{\contradiction}{{\hbox{%                                      % A contradiction hash symbol for QED.
    \setbox0=\hbox{$\mkern-3mu\times\mkern-3mu$}%
    \setbox1=\hbox to0pt{\hss$\times$\hss}%
    \copy0\raisebox{0.5\wd0}{\copy1}\raisebox{-0.5\wd0}{\box1}\box0
}}}
\newcommand{\ts}{\textsuperscript}                                        % Text-mode version of ^ (superscript).
\newcommand*{\mathcolor}{}                                                % math-mode version of \color.
\def\mathcolor#1#{\mathcoloraux{#1}}
\newcommand*{\mathcoloraux}[3]{%
	\protect\leavevmode
	\begingroup
		\color#1{#2}#3%
	\endgroup
}


% Vertical spacing commands
\newcommand{\aspace}{\vspace{2.5\baselineskip}}                           % Used for spaces between a question and its answer.
\newcommand{\qspace}{\vspace{1.5\baselineskip}}                           % Used for spaces between different questions.


% How to change QED symbols.
%\renewcommand{\qedsymbol}{$\square$\}                                    % Default square
%\renewcommand{\qedsymbol}{$\contradiction$}                              % Contradiction hash
%\renewcommand{\qedsymbol}{$\blacksquare$}                                % Black square
 %functions and commands

%--------------title stuff
%\usepackage[enable]{my-easy-todo} %local modified version of easy-todo

%pdftex needs to be almost the last package loaded
\usepackage[pdfauthor={Mohamar Rios Flores},%
            pdftitle={MATH 999 - Assignment 9},%
            pdffitwindow=false,%
            bookmarks=false,%
            unicode=true,%
            pdftex]{hyperref}
\usepackage{amsrefs} %AMS refs, needs to be loaded after hyperref


\title{MATH 999 -- Assignment 9}
\author{
  Rios Flores, Mohamar\\
  \texttt{mohamar.rios@gmail.com}%\\
%   {\small Collaborators:
%   Friesen, Alexis;
%   Mardon, Todd;
%   Vooys, Geoff;
%   Glin, Danny}
%
% \and
% Last, First\\
% \texttt{email@dom.com}
  }
%--------------\title stuff

\begin{document}
  %--------------header stuff
  %\pgfdeclarelayer{background}
  %\pgfdeclarelayer{foreground}
  %\pgfsetlayers{background,main,foreground}
  
  \maketitle
  %\tableofcontents
  %\section{some interesting words}
  %--------------\header stuff

\section*{Explanation of packages}

\section*{Examples}
  \begin{enumerate}

    \item
      Here are two different ways of writing an equation array. The \textit{IEEEeqnarray} version looks a little better and has slightly better functionality when it comes to line referencing and labels but it is more cumbersome to write than the \textit{align} version.

      \textbf{Note:} I'm using the \textit{enumitem} package for the \textit{enumerate} environment. It lets me manually define the numbering format and even give an explicit label.
      
      \begin{enumerate}[label = --\alph*--$>$]

      \item
        This is written using the align environment.

        \begin{align*}
          4xyzw
          &= 2\cdot2tu \\
          &\le 2\cdot(t^2+u^2)                    \tag{a remark in parentheses} \\
          &= 2\cdot((xy)^2+(zw)^2) \\
          &= 2\cdot(x^2y^2+z^2w^2)                \tag*{a remark without parentheses} \\
          &= 2x^2y^2+2z^2w^2 \\
          &\le ((x^2)^2+(y^2)^2)+((z^2)^2)+(w^2)^2) \\
          &= x^4+y^4+z^4+w^4                      \qedhere
        \end{align*}
      
      \item
        This is the same math written using the IEEEeqnarray environment.

        \textbf{Note:} I'm also calling \textit{renewcommand} on the QED symbol (the square), this lets me temporarily switch over to a contradiction symbol (defined in my preamble).

        \renewcommand{\qedsymbol}{$\contradiction$}

        \begin{IEEEeqnarray*}{rCl"s}
          4xyzw 
          &=& 2\cdot2tu             & \\
          &\le& 2\cdot(t^2+u^2)     & (a remark in parentheses) \\
          &=& 2\cdot((xy)^2+(zw)^2) & \\
          &=& 2\cdot(x^2y^2+z^2w^2) & a remark without parentheses \\
          &=& 2x^2y^2+2z^2w^2       & \\
          &\le& ((x^2)^2+(y^2)^2)+((z^2)^2)+(w^2)^2) & \\
          &=& x^4+y^4+z^4+w^4       & \qedhere
        \end{IEEEeqnarray*}

      \end{enumerate}

    \item
      Here are some examples of commutative diagrams. They're made using \textit{tikz} with the tikzlibrary \textit{cd}. 

      \begin{enumerate}
        \item[\textbf{Simple Diagram:}]

          \begin{equation*}
            \begin{tikzcd}
              A \arrow[r] \arrow[rd] & B \arrow[d] \arrow[rd] \\
                          & C \arrow[r] & D
            \end{tikzcd}
          \end{equation*}

        \item[\textbf{Complex Diagram:}]

          \begin{equation*}
            \begin{tikzcd}[row sep=scriptsize, column sep=scriptsize]
              & f^* E_V \arrow[dl] \arrow[rr, red] \arrow[dd, Rightarrow] & & E_V \arrow[dl] \arrow[dd] \\
              f^* E \arrow[rr, crossing over, blue] \arrow[dd] & & E \\
              & U \arrow[dl] \arrow[rr, bend left=20] \arrow[rr, bend right=20] & & V \arrow[dl] \\
              M \arrow[rr, "\varphi"] & & N \arrow[from=uu, crossing over]\\
            \end{tikzcd}
          \end{equation*}

      \end{enumerate}

  \end{enumerate}

  %--------------footer stuff
  \begin{bibdiv}
    \begin{biblist}
      \bib{Steen}{book}{
        title={Counterexamples in Topology},
        author={Steen, Lynn Arthur},
        author={Seebach, J. Arthur, Jr.},
        volume={18},
        year={1978},
        publisher={Springer}
      }
      \bib{Dasgupta}{book}{
        title = {Set Theory: With an Introduction to Real Point Sets},
        author = {Dasgupta, Abhijit},
        year = {2014},
        publisher = {Springer Science \& Business Media}
      }
      \bib{Landau}{book}{
        title = {Foundations of Analysis},
        volume={79},
        author = {Landau, Edmund},
        date = {1966},
        publisher = {American Mathematical Soc.},
        address = {Providence}
      }
      \bib{Suppes}{book}{
        title = {Axiomatic Set Theory},
        author = {Suppes, Patrick},
        date = {1972},
        publisher = {Dover Publications, Inc.},
        address = {New York}
      }
    \end{biblist}
  \end{bibdiv}
  %--------------\footer stuff
  
\end{document}

%http://tex.stackexchange.com/questions/40492/what-are-the-differences-between-align-equation-and-displaymath
